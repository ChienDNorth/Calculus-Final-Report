%%%%%%%%%%%%%%%%%%%%%%%%%%%%%%%%%%%%%%%%%
% Lachaise Assignment
% LaTeX Template
% Version 1.0 (26/6/2018)
%
% This template originates from:
% http://www.LaTeXTemplates.com
%
% Authors:
% Marion Lachaise & François Févotte
% Vel (vel@LaTeXTemplates.com)
%
% License:
% CC BY-NC-SA 3.0 (http://creativecommons.org/licenses/by-nc-sa/3.0/)
% 
%%%%%%%%%%%%%%%%%%%%%%%%%%%%%%%%%%%%%%%%%

%----------------------------------------------------------------------------------------
%	PACKAGES AND OTHER DOCUMENT CONFIGURATIONS
%----------------------------------------------------------------------------------------

\documentclass{report}
\usepackage{xcolor}
\usepackage{sectsty}
\usepackage{framed}
\usepackage{ntheorem}
\usepackage{physics}%加適當長度的竪綫 \eval{}(evaluate)

%%%%%%%%%%%%%%%%%%%%%%%%%%%%%%%%%%%%%%%%%
% Lachaise Assignment
% Structure Specification File
% Version 1.0 (26/6/2018)
%
% This template originates from:
% http://www.LaTeXTemplates.com
%
% Authors:
% Marion Lachaise & François Févotte
% Vel (vel@LaTeXTemplates.com)
%
% License:
% CC BY-NC-SA 3.0 (http://creativecommons.org/licenses/by-nc-sa/3.0/)
% 
%%%%%%%%%%%%%%%%%%%%%%%%%%%%%%%%%%%%%%%%%

%----------------------------------------------------------------------------------------
%	PACKAGES AND OTHER DOCUMENT CONFIGURATIONS
%----------------------------------------------------------------------------------------

\usepackage{amsmath,amsfonts,stmaryrd,amssymb} % Math packages

\usepackage{enumerate} % Custom item numbers for enumerations

\usepackage[ruled]{algorithm2e} % Algorithms

\usepackage[framemethod=tikz]{mdframed} % Allows defining custom boxed/framed environments

\usepackage{listings} % File listings, with syntax highlighting
\lstset{
	basicstyle=\ttfamily, % Typeset listings in monospace font
}

%----------------------------------------------------------------------------------------
%	DOCUMENT MARGINS
%----------------------------------------------------------------------------------------

\usepackage{geometry} % Required for adjusting page dimensions and margins

\geometry{
	paper=a4paper, % Paper size, change to letterpaper for US letter size
	top=2.5cm, % Top margin
	bottom=3cm, % Bottom margin
	left=2.5cm, % Left margin
	right=2.5cm, % Right margin
	headheight=14pt, % Header height
	footskip=1.5cm, % Space from the bottom margin to the baseline of the footer
	headsep=1.2cm, % Space from the top margin to the baseline of the header
	%showframe, % Uncomment to show how the type block is set on the page
}

%----------------------------------------------------------------------------------------
%	FONTS
%----------------------------------------------------------------------------------------

\usepackage[utf8]{inputenc} % Required for inputting international characters
\usepackage[T1]{fontenc} % Output font encoding for international characters

\usepackage{XCharter} % Use the XCharter fonts

%----------------------------------------------------------------------------------------
%	COMMAND LINE ENVIRONMENT
%----------------------------------------------------------------------------------------

% Usage:
% \begin{commandline}
%	\begin{verbatim}
%		$ ls
%		
%		Applications	Desktop	...
%	\end{verbatim}
% \end{commandline}

\mdfdefinestyle{commandline}{
	leftmargin=10pt,
	rightmargin=10pt,
	innerleftmargin=15pt,
	middlelinecolor=black!50!white,
	middlelinewidth=2pt,
	frametitlerule=false,
	backgroundcolor=black!5!white,
	frametitle={Command Line},
	frametitlefont={\normalfont\sffamily\color{white}\hspace{-1em}},
	frametitlebackgroundcolor=black!50!white,
	nobreak,
}

% Define a custom environment for command-line snapshots
\newenvironment{commandline}{
	\medskip
	\begin{mdframed}[style=commandline]
}{
	\end{mdframed}
	\medskip
}

%----------------------------------------------------------------------------------------
%	FILE CONTENTS ENVIRONMENT
%----------------------------------------------------------------------------------------

% Usage:
% \begin{file}[optional filename, defaults to "File"]
%	File contents, for example, with a listings environment
% \end{file}

\mdfdefinestyle{file}{
	innertopmargin=1.6\baselineskip,
	innerbottommargin=0.8\baselineskip,
	topline=false, bottomline=false,
	leftline=false, rightline=false,
	leftmargin=2cm,
	rightmargin=2cm,
	singleextra={%
		\draw[fill=black!10!white](P)++(0,-1.2em)rectangle(P-|O);
		\node[anchor=north west]
		at(P-|O){\ttfamily\mdfilename};
		%
		\def\l{3em}
		\draw(O-|P)++(-\l,0)--++(\l,\l)--(P)--(P-|O)--(O)--cycle;
		\draw(O-|P)++(-\l,0)--++(0,\l)--++(\l,0);
	},
	nobreak,
}

% Define a custom environment for file contents
\newenvironment{file}[1][File]{ % Set the default filename to "File"
	\medskip
	\newcommand{\mdfilename}{#1}
	\begin{mdframed}[style=file]
}{
	\end{mdframed}
	\medskip
}

%----------------------------------------------------------------------------------------
%	NUMBERED QUESTIONS ENVIRONMENT
%----------------------------------------------------------------------------------------

% Usage:
% \begin{question}[optional title]
%	Question contents
% \end{question}

\mdfdefinestyle{question}{
	innertopmargin=1.2\baselineskip,
	innerbottommargin=0.8\baselineskip,
	roundcorner=5pt,
	nobreak,
	singleextra={%
		\draw(P-|O)node[xshift=1em,anchor=west,fill=white,draw,rounded corners=5pt]{%
		Question \theQuestion\questionTitle};
	},
}

\newcounter{Question} % Stores the current question number that gets iterated with each new question

% Define a custom environment for numbered questions
\newenvironment{question}[1][\unskip]{
	\bigskip
	\stepcounter{Question}
	\newcommand{\questionTitle}{~#1}
	\begin{mdframed}[style=question]
}{
	\end{mdframed}
	\medskip
}

%----------------------------------------------------------------------------------------
%	WARNING TEXT ENVIRONMENT
%----------------------------------------------------------------------------------------

% Usage:
% \begin{warn}[optional title, defaults to "Warning:"]
%	Contents
% \end{warn}

\mdfdefinestyle{warning}{
	topline=false, bottomline=false,
	leftline=false, rightline=false,
	nobreak,
	singleextra={%
		\draw(P-|O)++(-0.5em,0)node(tmp1){};
		\draw(P-|O)++(0.5em,0)node(tmp2){};
		\fill[black,rotate around={45:(P-|O)}](tmp1)rectangle(tmp2);
		\node at(P-|O){\color{white}\scriptsize\bf !};
		\draw[very thick](P-|O)++(0,-1em)--(O);%--(O-|P);
	}
}

% Define a custom environment for warning text
\newenvironment{warn}[1][Warning:]{ % Set the default warning to "Warning:"
	\medskip
	\begin{mdframed}[style=warning]
		\noindent{\textbf{#1}}
}{
	\end{mdframed}
}

%----------------------------------------------------------------------------------------
%	INFORMATION ENVIRONMENT
%----------------------------------------------------------------------------------------

% Usage:
% \begin{info}[optional title, defaults to "Info:"]
% 	contents
% 	\end{info}

\mdfdefinestyle{info}{%
	topline=false, bottomline=false,
	leftline=false, rightline=false,
	nobreak,
	singleextra={%
		\fill[black](P-|O)circle[radius=0.4em];
		\node at(P-|O){\color{white}\scriptsize\bf i};
		\draw[very thick](P-|O)++(0,-0.8em)--(O);%--(O-|P);
	}
}

% Define a custom environment for information
\newenvironment{info}[1][Info:]{ % Set the default title to "Info:"
	\medskip
	\begin{mdframed}[style=info]
		\noindent{\textbf{#1}}
}{
	\end{mdframed}
}
 % Include the file specifying the document structure and custom commands

%----------------------------------------------------------------------------------------
%	ASSIGNMENT INFORMATION
%----------------------------------------------------------------------------------------

\title{MATH102 Calculus II \\1920 Final Report} % Title of the assignment
\author{Xue Ren Qianqian\\ \texttt{19098535-i011-0068}} % Author name and email address

\date{Macau University of Science and Technology \\\today} % University, school and/or department name(s) and a date

%----------------------------------------------------------------------------------------

\newtheorem{theorem}{THEOREM}
\newtheorem*{definition}{DEFINITION}

\begin{document}
\maketitle % Print the title

%----------------------------------------------------------------------------------------
%	Chapter 7
%----------------------------------------------------------------------------------------
\setcounter{chapter}{6}
\chapter{Transcendental  Functions}
\section{Inverse Functions and Their Derivatives} % Numbered section
\begin{quote}
	
	%------------------------------------------------
	
	\subsection{One-to-One Functions}
	\begin{quote}

		A function that has distinct values at distinct elements in its domain is called one-to-one.
		
		\begin{quote}
			\subsubsection{The Horizontal Line Test for One-to-One Functions}
			
			A function y = ƒ(x) is one-to-one if and only if its graph intersects each horizontal line at most once.
		    
		\end{quote}
	\end{quote}
	
	%------------------------------------------------
	
	%------------------------------------------------
	
	\subsection{Inverse Functions}
	\begin{quote}
		
		%----------------------

		\begin{definition}

			Suppose that ƒ is a one-to-one function on a domain D with range R.  The inverse function $f^{-1}$ is defined by
			$$f^{-1}(b) = a \quad if \quad f(a) = b$$
			The domain of $f^{-1}$ is R and the range of $f^{-1}$ is D.
		
		\end{definition}

		%----------------------

		\begin{quote}


			\begin{warn}[Notice:]
				The symbol $f^{-1}$ for the inverse of ƒ is read “ƒ inverse.” The “-1” in $f^{-1}$ is not an exponent.
			\end{warn}
			
			\subsubsection{How to Find Inverses}
			\begin{enumerate}
				\item Solve the equation y = ƒ(x) for x. This gives a formula x = $f^{-1}(y)$ where x is expressed as a function of y.
				\item Interchange x and y, obtaining a formula y = $f^{-1}(x)$ where $f^{-1}$ is expressed in the conventional format with x as the independent variable and y as the dependent v ariable.
			\end{enumerate}
			
		\end{quote}
	\end{quote}
	
	%------------------------------------------------
	%	7.1.3
	%------------------------------------------------
	
	\subsection{Derivatives of Inverses of Differentiable Functions}
	\begin{quote}
		
		If y = ƒ(x) has a horizontal tangent line at (a, ƒ(a)), then the inverse function $f^{-1}$ has a vertical tangent line at (ƒ(a), a), and this infinite slope implies that $f^{-1}$ is not differentiable at ƒ(a). Theorem 1 gives the conditions under which $f^{-1}$ is differentiable in its domain (which is the same as the range of ƒ).
		
		\begin{quote}
			\begin{theorem}[The Derivative Rule for Inverses]
				\mbox{}\par %Go to next row

				If ƒ has an interval I as domain and $f'(x)$ exists and is never zero on I, then $f'(x)$ is diffrentiable at every point in its domain (the range of f). The value of $(f^{-1})'$ at a point b in the domain of $f^{-1}$ is the reciprocal of the value of $f'$ at the point $a = f^{-1}(b)$:

				$$(f^{-1})'(b)=\frac{1}{f'(f^{-1}(b))}$$

				or

				$$\eval{\frac{df^{-1}}{dx}}_{x=b} = \frac{1}{\eval{\frac{df}{dx}}}_{x = f^{-1}(b)}$$

			\end{theorem}
		\end{quote}

	\end{quote}
	
\end{quote}

%----------------------------------------------------------------------------------------
%	7.4 Exponential Change and Separable Differential Equations
%----------------------------------------------------------------------------------------
\setcounter{section}{3}

\section{Exponential Change and Separable Differential Equations} % Numbered section
\begin{quote}

	\subsection{Exponential Change}
	\begin{quote}

		If the amount present at time $t = 0$ is called $y_0$, then we can find y as a function of t by solving the following initial value problem:
	
		\begin{quote}


			Diferential equation: \qquad $\frac{dy}{dt}=ky$ \par
			\setlength{\parskip}{2pt}
			Initial condition:\qquad $y = y_{0}$ \quad when \quad $t = 0$.\par
			\setlength{\parskip}{0pt}

			\begin{info} % Information block
				If y is positive and increasing, then $k>0$.\\
				If y is positive and decreasing, then $k<0$
			\end{info}

			\subsubsection{The solution of the initial value problem}
			$$\frac{dy}{dt}=ky, \qquad y(0)=y_{0}$$
			is
			$$y=y_{0}e^{kt}.$$

		\end{quote}
		

	\end{quote}

\end{quote}

%----------------------------------------------------------------------------------------
%	7.5 Exponential Change and Separable Differential Equations
%----------------------------------------------------------------------------------------

\section{Indeterminate Forms and L’Hôpital’s Rule}
\begin{quote}

\end{quote}

%----------------------------------------------------------------------------------------
%	7.6 Exponential Change and Separable Differential Equations
%----------------------------------------------------------------------------------------

\section{Inverse Trigonometric Functions}
\begin{quote}

\end{quote}

%----------------------------------------------------------------------------------------
%	Chapter 8
%----------------------------------------------------------------------------------------

\chapter{Transcendental  Functions}
\setcounter{section}{1}

%----------------------------------------------------------------------------------------
%	8.2
%----------------------------------------------------------------------------------------

\section{Integration by Parts}
\begin{quote}

\end{quote}

%----------------------------------------------------------------------------------------
%	8.3
%----------------------------------------------------------------------------------------

\section{Trigonometric Integrals}
\begin{quote}

\end{quote}

%----------------------------------------------------------------------------------------
%	8.4
%----------------------------------------------------------------------------------------

\section{Trigonometric Substitutions}
\begin{quote}

\end{quote}

%----------------------------------------------------------------------------------------
%	8.5
%----------------------------------------------------------------------------------------

\section{Integration of Rational Functions by Partial Fractions}
\begin{quote}

\end{quote}

%----------------------------------------------------------------------------------------
%	8.8
%----------------------------------------------------------------------------------------

\setcounter{section}{7}
\section{Integration of Rational Functions by Partial Fractions}
\begin{quote}

\end{quote}

%----------------------------------------------------------------------------------------
%	Chapter 10
%----------------------------------------------------------------------------------------

\setcounter{chapter}{9}
\chapter{Infinite Sequences and Series }

%----------------------------------------------------------------------------------------
%	10.1
%----------------------------------------------------------------------------------------

\section{Sequences}
\begin{quote}

\end{quote}

%----------------------------------------------------------------------------------------
%	10.2
%----------------------------------------------------------------------------------------

\section{Infinite Series }
\begin{quote}

\end{quote}

%----------------------------------------------------------------------------------------
%	10.3
%----------------------------------------------------------------------------------------

\section{The Integral Test }
\begin{quote}

\end{quote}

%----------------------------------------------------------------------------------------
%	10.4
%----------------------------------------------------------------------------------------

\section{Comparison Tests }
\begin{quote}

\end{quote}

%----------------------------------------------------------------------------------------
%	10.5
%----------------------------------------------------------------------------------------

\section{Absolute Convergence; The Ratio and Root Tests }
\begin{quote}

		\subsection{Absolute Convergence}
		\begin{quote}

			\begin{definition}
				A series $\sum a_n$ \textbf{converges absolutely} (is \textbf{absolutely convergent})  if the corresponding series of absolute values, $\sum \left | a_n \right |$ , converges.
			\end{definition}

			\begin{quote}

				\begin{theorem}
					If $\sum_{n=1}^{\infty} \left | a_n \right | $ converges, then $ \sum_{n=1}^{\infty} a_n $ converges.
				\end{theorem}

			\end{quote}

		\end{quote}

	\subsection{The Ratio Test}
	\begin{quote}

		\begin{quote}
		\begin{theorem}
			Let $ \sum a_n $ be any series and suppose that
			$$\lim_{n \to \infty} \left | \frac{a_{n+1}}{a_n} \right | = \rho $$
			Then \textbf{(a)} the series \textbf{converges absolutely} if $\rho < 1$, \textbf{(b)} the series \textbf{diverges} if $ \rho > 0 $ or $\rho$ is ininite, \textbf{(c)} the test is \textbf{inconclusive} if $\rho = 1$
		\end{theorem}
		\end{quote}

	\end{quote}

	\subsection{The Root Test}
	\begin{quote}

		\begin{quote}
			\begin{theorem}

				Let $\sum a_n$ be any series and suppose that
				$$\lim_{n \to \infty } \sqrt[n]{\left | a_n \right |} = \rho$$
				Then \textbf{(a)} the series \textbf{converges absolutely} if $\rho < 1$, \textbf{(b)} the series \textbf{diverges} if $ \rho > 0 $ or $\rho$ is ininite, \textbf{(c)} the test is \textbf{inconclusive} if $\rho = 1$
			
			\end{theorem}
		\end{quote}

	\end{quote}

\end{quote}

%----------------------------------------------------------------------------------------
%	10.6
%----------------------------------------------------------------------------------------

\section{Alternating Series and Conditional Convergence }
\begin{quote}

	A series in which the terms are alternately positive and negative is an alternating series. 

	\begin{quote}
		\begin{theorem}[The Alternating Series Test ]

			The series
			$$\sum_{n=1}^{\infty}(-1)^{n+1}u_n=u_1-u_2+u_3-u_4+\cdots$$
			converges if the following conditions are satisfied:

			\begin{enumerate}
				\item The $u_n$'s are all positive.
				\item The $u_n$ are eventually nonincreasing: $u_n \ge u_{n+1}$ for all $n\ge N$, for some integer N.
				\item $u_n \to 0$.
			\end{enumerate}

		\end{theorem}
	\end{quote}

	\begin{quote}
		\begin{theorem}[The Alternating Series Estimation Theorem]
			\mbox{}\par %Go to next row

			If the alternating series $\sum_{n=1}^{\infty} (-1)^{n+1} u_n$ satisies the three conditions of Theorem 15, then for $n \ge N$, 
			$$s_n = u_1 - u_2 \cdots + (-1)^{n+1}u_n$$
			approximates the sum L of the series with an error whose absolute value is less than $u_{n+1}$, the absolute value of the irst unused term. Furthermore, the sum L lies between any two successive partial sums $s_n$ and $s_{n+1}$, and the remainder, $L - s_n$, has the same sign as the irst unused term.

		\end{theorem}
	\end{quote}

	\subsection{Conditional Convergence}
	\begin{quote}
		
		\begin{definition}
		A series that is convergent but not absolutely convergent is called conditionally convergent.
		\end{definition}

	\end{quote}

	\subsection{Rearranging Series}
	\begin{quote}
		\begin{theorem}[The Rearrangement Theorem for Absolutely Convergent Series]
			\mbox{}\par

			If $\sum_{n=1}^{\infty}a_n$ converges absolutely, and $b_1, b_2, \cdots, b_n, \cdots$ is any arrangement of the sequence $\left \{ a_n \right \}$, then $\sum b_n$ converges absolutely and
			$$\sum_{n=1}^{\infty} b_n = \sum_{n=1}^{\infty}a_n$$

		\end{theorem}
	\end{quote}

	\subsection{Summary of Tests to Determine Convergence or Divergence}
	\begin{quote}

		Here is a summary of the tests we have considered.

		\begin{quote}
			
			\newpage
			
			\begin{enumerate}
				\item \textbf{The nth-term test for Divergence:} Unless $a_n \to 0$, the series diverges.
				\item \textbf{Geometric series:}                 $\sum ar^n$ converges if $\left | r \right |<1$; otherwise it diverges
				\item \textbf{p-series:}                         $\sum \frac{1}{n^p}$ converges if $p>1$; otherwise it diverges.
				\item \textbf{Series with nonnegative terms:}    Try the Integral Test or try comparing to a known series with the Direct Comparison Test or the Limit Comparison Test. Try the Ratio or Root Test.
				\item \textbf{Series with some negative terms:}  Does $\sum \left | a_n \right |$ converge by the Ratio or Root Test, or by another of the tests listed above? Remember, absolute convergence implies convergence.
				\item \textbf{Alternating series: }              $\sum a_n$ converges if the series satisfies the conditions of the Alternating Series Test.
			\end{enumerate}

			\begin{info} % Information block
				There are other tests we have not presented which are sometimes given in more advanced courses. 
			\end{info}
			
		\end{quote}


	\end{quote}

\end{quote}

%----------------------------------------------------------------------------------------
%	10.7
%----------------------------------------------------------------------------------------

\section{Power Series }
\begin{quote}

	\subsection{Power Series and Convergence}
	\begin{quote}

		\begin{definition}
			\textbf{A power series about x = 0} is a series of the form
			$$\sum_{n=0}^{\infty} c_n x^n  = c_0 + c_1x + \cdots$$
			\textbf{A pover series about x = a} is a series of the form
			$$\sum_{n=0}^{\infty}c_n (x-a)^n = c_0 + c_1 (x-a) + c_2 (x-a)^2 + \cdots + c_n (x-a)^n + \cdots$$
			in which the \textbf{center} a and the \textbf{coefficients} $c_0, c_1, c_2,\cdots , c_n,\cdots$ are constants.
		\end{definition}

		\begin{quote}
			\begin{theorem}[The Convergence Theorem for Power Series ]
				\mbox{}\par
				If the power series
				$\sum_{n=0}^{\infty} a_n x^n = a_0 + a_1x + a_2x^2 + \cdots $ converges at $x = c \neq 0$, then it converges absolutely for all x with $\left | x \right | < \left | c \right |$. If the series diverges at $x = d$. then it diverges for all x with $ \left | x \right | > \left | d \right |$.
			\end{theorem}
		\end{quote}

	\end{quote}
\end{quote}

%----------------------------------------------------------------------------------------
%	10.8
%----------------------------------------------------------------------------------------

\section{Taylor and Maclaurin Series }
\begin{quote}

\end{quote}

%----------------------------------------------------------------------------------------
%	Chapter 11
%----------------------------------------------------------------------------------------

\chapter{Parametric Equations and Polar Coordinates }
\begin{quote}

\end{quote}

%----------------------------------------------------------------------------------------
%	11.1
%----------------------------------------------------------------------------------------

\section{Parametrizations of Plane Curves }
\begin{quote}

\end{quote}

%----------------------------------------------------------------------------------------
%	11.2
%----------------------------------------------------------------------------------------

\section{Calculus with Parametric Curves }
\begin{quote}

\end{quote}

%----------------------------------------------------------------------------------------
%	11.3
%----------------------------------------------------------------------------------------

\section{Polar Coordinates }
\begin{quote}

\end{quote}

%----------------------------------------------------------------------------------------
%	11.4
%----------------------------------------------------------------------------------------

\section{Graphing Polar Coordinate Equations }
\begin{quote}

\end{quote}

%----------------------------------------------------------------------------------------
%	11.5
%----------------------------------------------------------------------------------------

\section{Areas and Lengths in Polar Coordinates  }
\begin{quote}

\end{quote}

\end{document}
